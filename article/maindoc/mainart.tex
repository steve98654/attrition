%
% File hicss51.tex
%
% Contact: Holm Smidt, hsmidt@hawaii.edu
%%
%%
%% Based on the style files for ACL 2015 by 
%% car@ir.hit.edu.cn, gdzhou@suda.edu.cn


\documentclass[10pt]{article}
\usepackage[letterpaper]{geometry}
\usepackage{hicss51}
\usepackage{times}
\usepackage[none]{hyphenat}
\usepackage{url}
\usepackage{latexsym}
\usepackage{minted}
\usepackage{graphicx}
\graphicspath{{images/}}

\newcommand{\sansserifformat}[1]{\fontfamily{cmss}{ #1}}%

%\setlength\titlebox{5cm}

% You can expand the titlebox if you need extra space
% to show all the authors. Please do not make the titlebox
% smaller than 5cm (the original size).


\title{An Explicative and Predictive Study of Employee Attrition}

\author{Nesreen El-rayes\\
  MTSM at NJIT \\
  {\underline{nde4@njit.edu}} \\\And
  Michael Smith\\
  MTSM at NJIT \\
  {\underline{mes6@njit.edu}}\\\And 
  Stephen Taylor\\
  MTSM at NJIT \\
  {\underline{smt@njit.edu} (Corr. Auth.)} \\}

\date{}

\begin{document}
\maketitle
\begin{abstract}
WRITE THIS IN THE END
\end{abstract}

\section{Introduction}
Building and maintaining a stable, productive, collaborative, and high-quality workforce is a primary concern 
for the majority of managing principals as success in this area tends to be a key factor contributing the 
overall firm prosperity [CITATIONS??].  Inevitably, all firms will experience employee attrition.  
Involuntary attrition is often the result of profitability and performance pressures, department or business 
line obsolescence, and mergers and acquisitions, among other factors [CITATIONS??].  In contrast,
voluntary attrition is driven predominately by employee concerns [CITATIONS].  Such considerations 
my focussed around, but not limited to, managerial direction, compensation and benefits, firm culture, 
firm desirability and location, promotion potential as well as non firm specific motivations, e.g. medical 
conditions or retirement. 
  
A central objective of the majority of human resource departments is to understand the root causes 
behind voluntary employee attrition and develop an associated mitigation strategy.  Effectively 
navigating such issues generally resulted in explicit positive monetary effects stemming from increased 
firm revenue and cost reductions manifested through the work retained highly performant employees. 
In addition, identifying and resolving issues found to be common to employee attrition often implicitly 
enhances firm culture and workplace desirably which in turn enables the recruitment of higher quality 
staff which further improve retention, firm operation, and business practices.  
We feel that the compounding effect nature of 
this employee attrition feedback loop on the overall success or failure of the firm is the essential motive 
for conducting a through investigation into this matter. [MORE CITATIONS FOR THIS PARAGRAPH]  

Traditionally, employee attrition and retention issues tend to examined through qualitative and anecdotal 
measures.  Specifically, human resources staff tend to conduct exit interviews after an employee provides 
a resignation notice in order to ascertain the motivations behind the decision to leave.  Although 
these conversations may be direct and candid, i.e. in the event an employee is leaving for a significantly 
more senior role or needs to change geographic location for family reasons, in actuality, human resources staff
encounter considerable difficulty discerning the employee's actual rationale.  By way of example, employees 
seldom offer negative criticism of management during exit interviews for fear of future personal retribution 
or inadvertent retaliation towards their close colleagues who will still remain at the firm.
These circumstances impact the employee attrition data aggregation and quality assurance process 
by making it cumbersome, at a minimum, which leads to additional difficulties determining which 
attrition issues should be of primary importance for management to resolve. In addition, employee 
attrition data is typically high confidential and only accessible to key stakeholders internally 
within a firm.  This fact has been a major impediment for the progression of academic research on 
this topic. [CITATIONS] 

Recently, internet based platforms such as GlassDoor and LinkedIn oriented towards working 
professionals have amassed large quantities of publicly available information related to individual 
employee resumes including employment history, frank reviews of firm culture, desirability, and management
as well as anonymous feedback.  Although this data often lacks attritional motivation information at 
the individual employee level, when combined with aggregate firm culture and management rankings, 
one may glean a number of insights into the collective behavior and motivations behind individual 
decisions to transition to a new employer. Our major aim is precisely in this vein.  More specifically, 
we conduct a quantitative data analysis of employee attrition motivations as well as develop 
predictive models that will enable human resources staff identify employees whose firm separation 
may be imminent.

We extend the work of \cite{glass} who examine employee attrition and retention issues based upon 
a collection of approximately five thousand anonymously submitted resumes to Glassdoor.  We provide 
SUMMARY OF DATA ANLAYSIS EXTENSIONS.  In addition, SUMMARY OF NEW MODELS. Lastly, we delineate 
future data acquisition, analysis and model development extensions that we seek to investigate 
in future work. 

This article is organized as follows 

\section{Formatting your paper}

All printed material, including text, illustrations, and charts, must be kept within a print area of 6-1/2 inches (16.51 cm) wide by 8-7/8 inches (22.51 cm) high. Do not write or print anything outside the print area. All text must be in a two-column format. Columns are to be 3 inches (7.85 cm) wide, with a 5.1/16 inch (0.81 cm) space between them. Text must be fully justified. \\
This formatting guideline provides the margins, placement, and print areas. If you hold it and your printed page up to the light, you can easily check your margins to see if your print area fits within the space allowed.

\section{Main title}

The main title (on the first page) should begin 1-3/8 inches (3.49 cm) from the top edge of the page, centered, and in Times 14-point, boldface type. Capitalize the first letter of nouns, pronouns, verbs, adjectives, and adverbs; do not capitalize articles, coordinate conjunctions, or prepositions (unless the title begins with such a word). Leave two 12-point blank lines after the title.

\section{Author name(s) and affiliation(s) }

Author names and affiliations must be included in the submitted Final Paper for Publication. Leave two 12-point blank lines after the author’s information. 

\section{Second and following pages}
\label{sect:pdf}

The second and following pages should begin 1.0 inch (2.54 cm) from the top edge. On all pages, the bottom margin should be 1-1/8 inches (2.86 cm) from the bottom edge of the page for 8.5 x 11-inch paper. (Letter-size paper)

\section{Type-style and fonts}
\label{sec:type-style}

Please note that {\em Times New Roman} is the preferred font for the text of you paper. \textbf{If you must use another font}, the following are considered base fonts.  You are encouraged to limit your font selections to Helvetica, Arial, and Symbol as needed. These fonts are automatically installed with the viewing software. 

\section{Page Numbers}

Please DO NOT include page numbers in your manuscript.

 

\section{Graphics/Images}

All images must be embedded in your document or included with your submission as individual source files. The type of graphics you include will affect the quality and size of your paper on the electronic document disc. In general, the use of vector graphics such as those produced by most presentation and drawing packages can be used without concern and is encouraged.

\begin{itemize}
\item Resolution: 600 dpi
\item Color Images: Bicubic Downsampling at 300dpi
\item Compression for Color Images: JPEG/Medium Quality
\item Grayscale Images: Bicubic Downsampling at 300dpi
\item Compression for Grayscale Images: JPEG/Medium Quality
\item Monochrome Images: Bicubic Downsampling at 600dpi
\item Compression for Monochrome Images: CCITT Group 4
\end{itemize}

If your paper contains many large images they will be down-sampled to reduce their size during the conversion process.  However the automated process used will not always produce the best image, and you are encouraged to perform this yourself on an image by image basis. The use of bitmapped images such as those produced when a photograph is scanned requires significant storage space and must be used with care.

\section{Main text}

Type your main text in 10-point Times, single-spaced. Do not use double-spacing. All paragraphs should be indented 1/4 inch (approximately 0.5 cm).  Be sure your text is fully justified—that is, flush left and flush right. Please do not place any additional blank lines between paragraphs. \\
\textbf{Figure and table captions} should be 9-point boldface Helvetica (or a similar sans-serif font).  Callouts should be 9-point non-boldface Helvetica. Initially capitalize only the first word of each figure caption and table title. Figures and tables must be numbered separately. For example: ``Figure 1. Database contexts'', ``Table 1. Input data''. Figure captions are to be centered below the figures. Table titles are to be centered above the tables.

% For one-column wide figures use
\begin{figure}[thb]
	% Use the relevant command to insert your figure file.
	% For example, with the graphicx package use
    \centering
	\includegraphics[trim={3cm 3cm 3cm 3cm}, clip,width=0.9\linewidth]{sample-image}
	% figure caption is below the figure
	\caption{Sample figure with caption.}
	\label{fig: sample-figure}       % Give a unique label
\end{figure}

\section{First-order headings}

For example, “1. Introduction”, should be Times 12-point boldface, initially capitalized, flush left, with one 12-point blank line before, and one blank line after. Use a period (“.”) after the heading number, not a colon. 

\subsection{Second-order headings}
 
As in this heading, they should be Times 11-point boldface, initially capitalized, flush left, with one blank line before, and one after. 

\subsubsection{Third-order headings. }

Third-order headings, as in this paragraph, are discouraged. However, if you must use them, use 10-point Times, boldface, initially capitalized, flush left, followed by a period and your text on the same line. 

\section{Footnotes}

      Use footnotes sparingly and place them at the bottom of the column on the page on which they are referenced. Use Times New Roman 8-point type, single-spaced. To help your readers, try to avoid using footnotes altogether and include necessary peripheral observations in the text (within parentheses, if you prefer, as in this sentence). 

% Fonts specification --- not shown as it doesn't exist in the Word document either. 

%\section{Fonts}

%A summary of fonts is provided in Table \ref{tab: fonts}. 

%\begin{table}[thb]
%\centering
%\caption{\label{font-table} Font guide. \vskip 3pt }
%\label{tab: fonts}
%\begin{tabular}{l|rl}
%\hline \bf Type of Text & \bf Font Size & \bf Style \\ \hline
%paper title & 14 pt &  \bf bold \\
%authors & 10 pt &  \underline{email} underlined \\
%abstract title & 12 pt &  \bf bold\\
%abstract text & 10 pt &  \it italic\\
%section titles & 12 pt & \bf bold \\
%subsection titles & 11 pt & \bf bold \\
%document text & 10 pt  & \\
%captions & 9 pt & \sansserifformat{\captionsize sans-serif, \bf bold} \\
%bibliography & 9 pt & \\
%footnotes & 8 pt & \\
%\hline
%\end{tabular}
%\end{table}


\section{References} 

List and number all bibliographical references in 9-point Times, single-spaced, at the end of your paper. When referenced in the text, enclose the citation number in square brackets, for example \cite{Jones2015,Smith2015} and \cite{Smith2015}. Where appropriate, include the name(s) of editors of referenced books.

% if added before the last page, this command can help balancing columns
%\addtolength{\textheight}{-.2cm} 

%Bibliography 
\bibliographystyle{ieeetr}
\bibliography{sample}


\end{document}
