%
% File hicss51.tex
%
% Contact: Holm Smidt, hsmidt@hawaii.edu
%%
%%
%% Based on the style files for ACL 2015 by 
%% car@ir.hit.edu.cn, gdzhou@suda.edu.cn


\documentclass[10pt]{article}
\usepackage[letterpaper]{geometry}
\usepackage{hicss51}
\usepackage{times}
\usepackage[none]{hyphenat}
\usepackage{url}
\usepackage{latexsym}
\usepackage{minted}
\usepackage{graphicx}
\graphicspath{{images/}}

\newcommand{\sansserifformat}[1]{\fontfamily{cmss}{ #1}}%

%\setlength\titlebox{5cm}

% You can expand the titlebox if you need extra space
% to show all the authors. Please do not make the titlebox
% smaller than 5cm (the original size).


\title{An Explicative and Predictive Study of Employee Attrition}

\author{Nesreen El-rayes\\
  MTSM at NJIT \\
  {\underline{nde4@njit.edu}} \\\And
  Michael Smith\\
  MTSM at NJIT \\
  {\underline{mes6@njit.edu}}\\\And 
  Stephen Taylor\\
  MTSM at NJIT \\
  {\underline{smt@njit.edu} (Corr. Auth.)} \\}

\date{}

\begin{document}
\maketitle
\begin{abstract}
WRITE THIS IN THE END
\end{abstract}

\section{Introduction}
Building and maintaining a stable, productive, collaborative, and high-quality workforce is a primary concern 
for the majority of managing principals as success in this area tends to be a key factor contributing the 
overall firm prosperity [CITATIONS??].  Inevitably, all firms will experience employee attrition.  
Involuntary attrition is often the result of profitability and performance pressures, department or business 
line obsolescence, and mergers and acquisitions, among other factors [CITATIONS??].  In contrast,
voluntary attrition is driven predominately by employee concerns [CITATIONS].  Such considerations 
my focussed around, but not limited to, managerial direction, compensation and benefits, firm culture, 
firm desirability and location, promotion potential as well as non firm specific motivations, e.g. medical 
conditions or retirement. 
  
A central objective of the majority of human resource departments is to understand the root causes 
behind voluntary employee attrition and develop an associated mitigation strategy.  Effectively 
navigating such issues generally resulted in explicit positive monetary effects stemming from increased 
firm revenue and cost reductions manifested through the work retained highly performant employees. 
In addition, identifying and resolving issues found to be common to employee attrition often implicitly 
enhances firm culture and workplace desirably which in turn enables the recruitment of higher quality 
staff which further improve retention, firm operation, and business practices.  
We feel that the compounding effect nature of 
this employee attrition feedback loop on the overall success or failure of the firm is the essential motive 
for conducting a through investigation into this matter. [MORE CITATIONS FOR THIS PARAGRAPH]  

Traditionally, employee attrition and retention issues tend to examined through qualitative and anecdotal 
measures.  Specifically, human resources staff tend to conduct exit interviews after an employee provides 
a resignation notice in order to ascertain the motivations behind the decision to leave.  Although 
these conversations may be direct and candid, i.e. in the event an employee is leaving for a significantly 
more senior role or needs to change geographic location for family reasons, in actuality, human resources staff
encounter considerable difficulty discerning the employee's actual rationale.  By way of example, employees 
seldom offer negative criticism of management during exit interviews for fear of future personal retribution 
or inadvertent retaliation towards their close colleagues who will still remain at the firm.
These circumstances impact the employee attrition data aggregation and quality assurance process 
by making it cumbersome, at a minimum, which leads to additional difficulties determining which 
attrition issues should be of primary importance for management to resolve. In addition, employee 
attrition data is typically high confidential and only accessible to key stakeholders internally 
within a firm.  This fact has been a major impediment for the progression of academic research on 
this topic. [CITATIONS] 

Recently, internet based platforms such as GlassDoor and LinkedIn oriented towards working 
professionals have amassed large quantities of publicly available information related to individual 
employee resumes including employment history, frank reviews of firm culture, desirability, and management
as well as anonymous feedback.  Although this data often lacks attritional motivation information at 
the individual employee level, when combined with aggregate firm culture and management rankings, 
one may glean a number of insights into the collective behavior and motivations behind individual 
decisions to transition to a new employer. Our major aim is precisely in this vein.  More specifically, 
we conduct a quantitative data analysis of employee attrition motivations as well as develop 
predictive models that will enable human resources staff identify employees whose firm separation 
may be imminent.

We extend the work of \cite{Smart2016} who examine employee attrition and retention issues based upon 
a collection of approximately five thousand anonymously submitted resumes to Glassdoor.  We provide 
SUMMARY OF DATA ANLAYSIS EXTENSIONS.  In addition, SUMMARY OF NEW MODELS. Lastly, we delineate 
future data acquisition, analysis and model development extensions that we seek to investigate 
in future work. 

This article is organized as follows ... TYPE WHEN FINISHED 

\section{Data Description and Feature Engineering}

We first turn to describing the content extracted from a collection of 
employee resumes that will form the basis for the subsequent attrition studies. 
Next, we provide a variety of summary statistics of this information that 
are relevant for the design of latter predictive models.  Then we discuss 
our data normalization process and 
several features constructed from this original data which will be utilized 
as input into these models.
In addition, we outline limitations of this dataset and ideas for improvement  
in future work.

\subsection{Data Source Description}

We worked in conjunction with the authors of \cite{Smart2016} to obtain 
a collection of 5550 examples of employee job transitions between 
2007 and 2016 which were sourced from an extensive proprietary database of 
resumes shared anonymously though Glassdoor's platform.  A job transition 
is define to be any instance of an employee listing a new role on their 
resume which may be associated with the present or original or a new employer
designated as internal and external moves.  Internal moves are typically 
significant in the sense of the employee either changing roles or 
being promoted within an organization as opposed to being reassigned 
within a current team. External moves are of interest for our 
attrition studies as in this situation employees leave their original 
firm entirely.
We summarize several salient features of the dataset construction process 
and expound upon details relevant to model development below; we refer the
reader to \cite{Smart2016} for a complete description of the data source. 




\subsection{Summary Statistics and Feature Engineering}

\section{Exploratory Insights}

Sum. 

\subsection{Idea 1}


\subsection{Idea 2}

\section{Towards an Attrition Model}

adf

\subsection{Linear Review}


\subsection{Logistic Work}


\subsection{Binary Classifier}

\subsection{Model Performance Comparison}

\section{Conclusions and Extensions}


Here we describe several ideas to pursue as part of this study

Mention limitations here (same as in glassdoor article and others that we find) ... then 
mention that our extensions will overcome these limitations 

-- company specific analytics 
-- linkedin, full glassdoor dataset 
-- outlier determination may serve as a glassdoor fake review filter 


% For one-column wide figures use
%\begin{figure}[thb]
	% Use the relevant command to insert your figure file.
	% For example, with the graphicx package use
%    \centering
%	\includegraphics[trim={3cm 3cm 3cm 3cm}, clip,width=0.9\linewidth]{sample-image}
	% figure caption is below the figure
%	\caption{Sample figure with caption.}
%	\label{fig: sample-figure}       % Give a unique label
%\end{figure}

\textbf{Acknowledgements:} The authors would like to ack... for sharing data FINISH WHEN DONE

% if added before the last page, this command can help balancing columns
%\addtolength{\textheight}{-.2cm} 

%Bibliography 
\bibliographystyle{ieeetr}
\bibliography{sample}


\end{document}
