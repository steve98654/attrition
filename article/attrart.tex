\documentclass{amsart}[12pt]
\usepackage{amssymb}
\usepackage{graphics}
\usepackage{graphicx}
\usepackage{verbatim}
\usepackage{amssymb, epsfig}
\usepackage[leqno]{amsmath}
\usepackage{eurosym}
\usepackage{listings}

\usepackage{float}

\def\P{\mathbb{P}}
\def\Q{\mathbb{Q}}
\def\E{\mathbb{E}}

\newtheorem{theorem}{Theorem}[section]
\newtheorem{lemma}[theorem]{Lemma}
\newtheorem{proposition}[theorem]{Proposition}
\newtheorem{example}[theorem]{Example}
\newtheorem{definition}[theorem]{Definition}
\newtheorem{corollary}[theorem]{Corollary}
\newtheorem{remark}[theorem]{Remark}
\newtheorem{assumption}[theorem]{Assumption}

\usepackage[a4paper,margin=3cm]{geometry}
\usepackage{authblk}

%\title{Bootstrapping Based Asset Allocation}
%\author{Wolfgang H\"{a}rdle, 
\title{Notes on GlassDoor Employee Attrition Study}
%\author{Stephen Taylor, Jan Vecer}
%\email{steve98654@gmail.com}
%\author[1]{Stephen Taylor}
%\author[2]{Jan Vecer}
%\affil[1]{New Jersey Institute of Technology, Martin Tuchman School of Management,
%3000 Central Avenue Building (CAB), Newark, New Jersey 07102, USA}
%\affil[2]{Charles University, Department of Probability and Mathematical Statistics, Sokolovska 83, 18675 Praha 8, Czech Republic}
%\date{March 4, 2018}

\begin{document}

\maketitle

\begin{abstract}
  We ...
\end{abstract}

\section{Introduction and Overview}

-- high level summary 

-- literature review

-- discuss why data is diff to find here 

-- organization of article 

\section{Data Description and Summary} 

In this section we ... 

\subsection{Data Description}

\subsection{Data Summary Insights}

\section{Towards an Attrition Model} 

\subsection{Variable Selection and Feature Engineering} 

\subsection{Model Creation} 

\subsection{Performance Comparison}

\section{Conclusions and Extensions} 

\subsection{Review and Limitations}


\subsection{Future Extensions }

Here we describe several ideas to pursue as part of this study

Mention limitations here (same as in glassdoor article and others that we find) ... then 
mention that our extensions will overcome these limitations 

-- company specific analytics 
-- linkedin, full glassdoor dataset 
-- outlier determination may serve as a glassdoor fake review filter 


\begin{thebibliography}{100}
    \bibitem{Sma} Smart, Morgan and Chamberlain, Andrew (2016). Why Do Workers Quit? The Factors 
        that Predict Employee Turnover. \emph{Glassdoor Research Report Whitepaper}.  
\end{thebibliography}


\section{IDEAS TO PURSUE FOR THIS PROJECT}

\begin{enumerate}
    \item Old to new industry ... matrix here 
    \item Let's focus on Leaving a company, not just different role in same company 
    \item Look at rankings from old company to new one; what can we say overall about 
         the characteristics of these companies?  What factors did we see the most signi
         change in? Make scatter plots/hists here 
    \item What other summary statistics are relevant that go beyond what is already in 
          the glassdoor article? 
    \item Do a brief literature review.  What has been done in this area already?
          Summarize Glassdoor as part of this ... describe the uniqueness of this dataset. 
          Mention it would be of interest to expand ... mine LinkedIn as well to do a 
          more thorough study later. 
    \item do reasons vary by length of job??? 
    \item Look at people who made a major change (e.g. shifted geographic regions. ... more than 500 miles 
          away ... is motivation any different for these people?)
    \item Look at how old vs new ranking variable scatter plots differ from the y=x line, 
        e.g. if employees change, which variables to we find also change significantly?  
          Do some change always in the upwards direction? Linear reg. essentially tries 
          to understand to what extent is this line differs from 1, e.g. if it is 1 for 
          a given variable, then this variable has no influence ... can we develop a 
          better variable influence measure here? 
    \item Look at people who switched industries ... any different motivation here? 
    \item One interesting feature is that employees seldom seem to stay at a company that has the 
          same size as their old one.  Lots of shirts from large to small or vice versa.  Perhaps 
          make this more quantitative? 
    \item  check if job-len corresponds to start,end date difference 
    \item Any info associated with geographic studies?  Large vs small cities? 
    \item  patterns that stem from either age or employment tenure? 
    \item binary model classifer, ROC curve etc. 
    \item impl their original regression model ... test 
    \item  industry specific studies 
    \item Distribution of dates that changes were made ... do we see a pattern on times of the 
          year that moves occurred 
\end{enumerate}

\end{document}

